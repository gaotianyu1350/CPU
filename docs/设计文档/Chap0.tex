\chapter{文档说明}

本文档是NonExist组软工计原联合项目的设计文档,作为设计文档,将会尽可能详细的覆盖到所有的设计方面和设计细节。

但是,设计文档呈现的是最终版CPU,因此每个模块此时都已经经历了无数次蜕变,在这个过程中,每个模块的功能越来越多,也越来越复杂。本文档呈现的是最终版的设计,缺少了循序渐进的过程,因此读者初读起来可能感到困难,不建议将此文档作为前期主要参考文档。

本文档正确的使用方式是,开发初期通读全文,站在更高的层面上俯视了解整个项目的设计;开发后期精读细节,实现和完善具体的功能。

设计文档将项目分成了以下部分:

    \begin{enumerate}
        \itembf{指令流水}:本阶段实现CPU五级流水线及绝大部分基本指令。
        \itembf{控制模块}:本阶段实现协处理器和异常处理
        \itembf{内存管理}:本阶段实现内存管理。
        \itembf{外设连接}:本阶段完成外设连接。
        \itembf{仿真调试}:本阶段加入模拟的硬件模块完成仿真调试。
    \end{enumerate}


设计文档每个章节遵从以下介绍流程:

    \begin{enumerate}
        \itembf{简介概述}: 简单介绍所实现元件的功能。
        \itembf{接口定义}: 实现的接口及其含义。
        \itembf{设计细节}: 详细而完善的设计思路与细节。
    \end{enumerate}

希望本文档能给读者带来裨益。

