% !Mode:: "TeX:UTF-8"
\documentclass{report}
\usepackage[hyperref, UTF8]{ctex}
\usepackage[dvipsnames]{xcolor}
\usepackage{amsmath}
\usepackage{amssymb}
\usepackage{amsfonts}
\usepackage{listings}
\usepackage{pgfplotstable}
\usepackage{graphicx,float,wrapfig}
\usepackage{pgfplots}
\usepackage{fontspec}
\usepackage{hyperref}
\usepackage{booktabs} % 表格上的不同横线
\usepackage{ifthen}
\usepackage[top = 1in, bottom = 1in, left = 1in, right = 1in]{geometry}

\setmonofont[Mapping={}]{Monaco}  %英文引号之类的正常显示,相当于设置英文字体
\setsansfont{Monaco} %设置英文字体 Monaco, Consolas,  Fantasque Sans Mono

\newcommand{\image}[3][3in]{
    \begin{figure}[H]
        \centering\includegraphics[width=#1]{images/#2}
        \caption{#3}
    \end{figure}
}

\newcommand{\miniimage}[3][2.5in] {
    \begin{minipage}[H]{0.4\linewidth}
        \centering\includegraphics[width=#1]{images/#2}
        \caption{#3}
    \end{minipage}
}

\newcommand{\itembf}[1]{\item \textbf{#1}}

\newcommand{\tabletwo}[2] {
    \begin{table}[H] \centering
    \begin{tabular}{ c c }
    \toprule
    \textbf{#1} & \textbf{#2} \\
    \midrule
}

\newcommand{\tabletwoL}[2] {
    \begin{table}[H] \centering
    \begin{tabular}{ l l }
    \toprule
    \textbf{#1} & \textbf{#2} \\
    \midrule
}

\newcommand{\tablethree}[3] {
    \begin{table}[H] \centering
    \begin{tabular}{ c c c }
    \toprule
    \textbf{#1} & \textbf{#2} & \textbf{#3} \\
    \midrule
}

\newcommand{\tablethreeL}[3] {
    \begin{table}[H] \centering
    \begin{tabular}{ l l l }
    \toprule
    \textbf{#1} & \textbf{#2} & \textbf{#3} \\
    \midrule
}

\newcommand{\tablefour}[4] {
    \begin{table}[H] \centering
    \begin{tabular}{ c c c c }
    \toprule
    \textbf{#1} & \textbf{#2} & \textbf{#3} & \textbf{#4} \\
    \midrule
}

\newcommand{\tablefourL}[4] {
    \begin{table}[H] \centering
    \begin{tabular}{ l l l l }
    \toprule
    \textbf{#1} & \textbf{#2} & \textbf{#3} & \textbf{#4} \\
    \midrule
}

\newcommand{\tablefive}[5] {
    \begin{table}[H] \centering
    \begin{tabular}{ c c c c c }
    \toprule
    \textbf{#1} & \textbf{#2} & \textbf{#3} & \textbf{#4} & \textbf{#5} \\
    \midrule
}

\newcommand{\tablefiveL}[5] {
    \begin{table}[H] \centering
    \begin{tabular}{ l l l l l }
    \toprule
    \textbf{#1} & \textbf{#2} & \textbf{#3} & \textbf{#4} & \textbf{#5} \\
    \midrule
}

\newcommand{\tablesix}[6] {
    \begin{table}[H] \centering
    \begin{tabular}{ c c c c c c }
    \toprule
    \textbf{#1} & \textbf{#2} & \textbf{#3} & \textbf{#4} & \textbf{#5} & \textbf{#6} \\
    \midrule
}

\newcommand{\tablesixL}[6] {
    \begin{table}[H] \centering
    \begin{tabular}{ l l l l l l }
    \toprule
    \textbf{#1} & \textbf{#2} & \textbf{#3} & \textbf{#4} & \textbf{#5} & \textbf{#6} \\
    \midrule
}

\newcommand{\tableend} {
    \bottomrule
    \end{tabular}
    \end{table}
}

\newcommand{\chuhao}{\fontsize{42pt}{\baselineskip}\selectfont}
\newcommand{\xiaochuhao}{\fontsize{36pt}{\baselineskip}\selectfont}
\newcommand{\yihao}{\fontsize{28pt}{\baselineskip}\selectfont}
\newcommand{\erhao}{\fontsize{21pt}{\baselineskip}\selectfont}
\newcommand{\xiaoerhao}{\fontsize{18pt}{\baselineskip}\selectfont}
\newcommand{\sanhao}{\fontsize{15.75pt}{\baselineskip}\selectfont}
\newcommand{\sihao}{\fontsize{14pt}{\baselineskip}\selectfont}
\newcommand{\xiaosihao}{\fontsize{12pt}{\baselineskip}\selectfont}
\newcommand{\wuhao}{\fontsize{10.5pt}{\baselineskip}\selectfont}
\newcommand{\xiaowuhao}{\fontsize{9pt}{\baselineskip}\selectfont}
\newcommand{\liuhao}{\fontsize{7.875pt}{\baselineskip}\selectfont}
\newcommand{\shibahao}{\fontsize{18pt}{\baselineskip}\selectfont}
\newcommand{\shisihao}{\fontsize{14pt}{\baselineskip}\selectfont}
\newcommand{\qihao}{\fontsize{5.25pt}{\baselineskip}\selectfont}

\definecolor{mygreen}{rgb}{0,0.6,0}
\definecolor{mygray}{rgb}{0.5,0.5,0.5}
\definecolor{mymauve}{rgb}{0.58,0,0.82}
\lstset{ %
    backgroundcolor=\color{white},   % choose the background color
    basicstyle=\footnotesize\ttfamily,        % size of fonts used for the code
    columns=fullflexible,
    breaklines=true,                 % automatic line breaking only at whitespace
    captionpos=b,                    % sets the caption-position to bottom
    tabsize=4,
    backgroundcolor=\color[RGB]{245,245,244},            % 设定背景颜色
    commentstyle=\color{mygreen},    % comment style
    escapeinside={\%*}{*)},          % if you want to add LaTeX within your code
    keywordstyle=\color{blue},       % keyword style
    stringstyle=\color{mymauve}\ttfamily,     % string literal style
    showstringspaces=false,                % 不显示字符串中的空格
    frame=none,
    rulesepcolor=\color{red!20!green!20!blue!20},
    % identifierstyle=\color{red},
    language=python,
}

% 设置hyperlink的颜色
\newcommand\myshade{85}
\colorlet{mylinkcolor}{violet}
\colorlet{mycitecolor}{YellowOrange}
\colorlet{myurlcolor}{Aquamarine}

\hypersetup{
  linkcolor  = mylinkcolor!\myshade!black,
  citecolor  = mycitecolor!\myshade!black,
  urlcolor   = myurlcolor!\myshade!black,
  colorlinks = true,
}

\title{\yihao{软工计原联合项目} \\ \erhao{设计文档}}
\date{}
\author{NonExist组\\张钰晖,杨一滨,周正平}
\begin{document}
\maketitle
\clearpage
\tableofcontents

\chapter{文档说明}

本文档是NonExist组软工计原联合项目的设计文档,作为设计文档,将会尽可能详细的覆盖到所有的设计方面和设计细节。

但是,设计文档呈现的是最终版CPU,因此每个模块此时都已经经历了无数次蜕变,在这个过程中,每个模块的功能越来越多,也越来越复杂。本文档呈现的是最终版的设计,缺少了循序渐进的过程,因此读者初读起来可能感到困难,不建议将此文档作为前期主要参考文档。

本文档正确的使用方式是,开发初期通读全文,站在更高的层面上俯视了解整个项目的设计;开发后期精读细节,实现和完善具体的功能。

设计文档将项目分成了以下部分:

    \begin{enumerate}
        \itembf{指令流水}:本阶段实现CPU五级流水线及绝大部分基本指令。
        \itembf{控制模块}:本阶段实现协处理器和异常处理
        \itembf{内存管理}:本阶段实现内存管理。
        \itembf{外设连接}:本阶段完成外设连接。
        \itembf{仿真调试}:本阶段加入模拟的硬件模块完成仿真调试。
    \end{enumerate}


设计文档每个章节遵从以下介绍流程:

    \begin{enumerate}
        \itembf{简介概述}: 简单介绍所实现元件的功能。
        \itembf{接口定义}: 实现的接口及其含义。
        \itembf{设计细节}: 详细而完善的设计思路与细节。
    \end{enumerate}

希望本文档能给读者带来裨益。



\chapter{系统总览}

以下给出本次工程的系统总览(其中为简洁起见,省略了部分连接线路):

\image[6.3in]{outline}{系统总览}

从图中可以看出,系统中包括openmips和thinpad\_top共2个较为重要的层级,以下为简要说明:

    \paragraph{openmips} openmips集成了CPU的核心部分,包括五级流水线、MMU、寄存器、CP0等。
    \paragraph{thinpad\_top} thinpad\_top为与开发板衔接的顶层模块,它在openmips之外还集成了对ROM、RAM、Flash、串口、VGA等多种外设的控制逻辑。


\chapter{开发周期}

我们将开发周期分成5个Sprint,具体展开每个开发周期应该做什么,通过这样详细的拆分,希望读者可以能尽快走出迷茫期,对整个项目有一个大概的认知。

\section{Sprint 1}

Sprint 1的持续时间为Week 2 - Week 4,本阶段我们刚刚接触项目,并加上十一国庆节,在这个阶段,我们做了:

    \begin{enumerate}
        \itembf{阅读文献}: 打印阅读《自己动手写CPU》和学长文档。《自己动手写CPU》是一本极其重要的书,作者清晰的思路带领读者一步一步的从零开始搭建出一个流水线框架,因此建议每人一本,快速阅读。
        \itembf{配置环境}: 学习软工平台和Gitlab,安装Vivado,配置Ubuntu云服务器。由于我们三个人的电脑均不是Ubuntu,故我们选择购买了一台最低配的云服务器作为我们Ubuntu的开发环境,从而实现安装mips编译器,编译功能测例、ucore等等,云服务器给我们提供了完全相同资源共享的环境,且上传下载简单。
        \itembf{明确需求}: 事实上,在这个阶段,尽管我们读了《自己动手写CPU》和学长文档,但是因为知识面远远不够,我们对项目的认知还是较为模糊,无法进行明确分工。
    \end{enumerate}

    \image[5in]{cpu}{神书《自己动手写CPU》}

从Milestone也可以看出,这段时期我们并没有明显的开发痕迹,主要是阅读资料。

\section{Sprint 2}

Sprint 2的持续时间为Week 5 - Week 7。在这个阶段,我们做了:

    \begin{enumerate}
        \itembf{搭建流水线框架}:实现MIPS标准五级流水线,只支持ori指令。
        \itembf{增加各类指令}:增加逻辑运算指令、算数运算指令、分支跳转指令、访存指令等ucore运行所需的大部分指令。
        \itembf{设计数据通路}:解决流水线竞争与冒险,数据前推,增加控制器实现流水线暂停功能。
        \itembf{单一指令测试}:搭建SOPC,对实现的每条指令进行简单的测试,检查实现是否正确。
    \end{enumerate}

这个阶段是后面所有阶段的基础,推荐的方法是快速阅读与实现《自己动手写CPU》相关内容,本阶段可以分工,每个成员实现一章的指令,但是每个成员都要读相关章节,这样才能保证所有人能对系统有一个大概的认知。

\section{Sprint 3}

Sprint 3的持续时间为Week 8 - Week 10。在这个阶段,我们做了:
    
    \begin{enumerate}
        \itembf{异常处理}:实现CP0协处理器与异常处理,对其进行相应的的测试。
        \itembf{内存管理}:阅读TLB、MMU相关文献,理解TLB、MMU的基本原理,实现TLB、MMU从而进行内存管理。
        \itembf{外设连接}:了解各种外设(Flash、RAM、串口)的使用方法,写代码对其进行简单测试
        \itembf{仿真操统}:在搭建的云服务器上,编译ucore,并使用qemu进行仿真,了解我们的终极目标是什么。
    \end{enumerate}

这个阶段已经算是进阶了,已经对CPU基本的框架有了相应的了解,本阶段可以分工,异常处理由一个人完成,推荐的方法是快速阅读与实现《自己动手写CPU》相关内容,之后《自己动手写CPU》这本书便完成了它的使命,后期已经没有可以参考直接使用的东西了,最多可以参考该书了解一些相关概念!内存管理推荐一个人完成,上网查阅理解TLB、MMU的基本概念,或者向学长老师要一份往年的计原很靠后的PPT,从而实现内存管理模块。外设连接推荐一个人完成,对提供的顶层代码写一些验证性代码,理解外设的使用方法,掌握外设的时序关系,能阅读相应的英文外设文档。分工后组员一定要进行相互讲解,review对方的代码,进一步加深对系统的理解。

\section{Sprint 4}

Sprint 4的持续时间为Week 11 - Week 13。在这个阶段,我们做了:
    
    \begin{enumerate}
        \itembf{调试功能测例}:功能测例内含九十余条指令的测试代码,编译后烧录至RAM中便可以测试CPU相关指令是否实现正确,功能测例对指令实现要求高,对外设实现要求低。我们先仿真通过了所有功能测例,又使用硬件通过了所有功能测例。
        \itembf{调试监控程序}:监控程序对指令实现要求低,对外设实现要求高,调试需要实现ROM、Flash和串口,实现boot过程。
    \end{enumerate}

进入这个阶段已经没有什么文档可以参考了,因为每个人的问题都不一样,推荐的方法是根据调试文档中的调试技巧,定位问题所在,查阅学长文档和网上的资料,三人共同讨论分析问题、解决问题。

\section{Sprint 5}

Sprint 4的持续时间为Week 14 - Week 16。在这个阶段,我们做了:
    
    \begin{enumerate}
        \itembf{TLB功能测例}:依据功能测例的基本框架,对TLBWR和TLBWI两条指令撰写功能测例。
        \itembf{调试ucore}:ucore对指令实现要求高,对外设实现要求高,但由于功能测例和监控程序都已进行了充分的测试,故本阶段调试较为顺利,仅耗时2天。
        \itembf{增加外设}:增加显存模块,增加VGA输出模块,实现图像输出功能。
        \itembf{文档撰写}:总结经验教训,总结开发心得。
    \end{enumerate}

如果前面每个组员都完成了自己的任务,对系统有一个清晰的理解,每项功能都经历了严格的测试,积累了大量的经验,本阶段将不再那么痛苦,但是如果之前阶段得过且过,本阶段可能会遇到无数的问题,因此之前的阶段一定要认真开发,不要划水!







\chapter{控制模块}

\section{ctrl}

    \subsection{简介概述}
    a

    \subsection{接口定义}
    a

    \subsection{设计细节}
    a

\section{cp0\_reg}

    \subsection{简介概述}
    a

    \subsection{接口定义}
    a

    \subsection{设计细节}
    a



\chapter{输出调试}

\section{简介概述}

如果读者已经把串口调试通过,可以通过串口收发信息,那么可以通过修改软件直接将数据发送至串口查看硬件信息。

听起来非常棒,就像是调试软件中可以向屏幕print信息一样,但事实上,现实并没有那么美好。

(1)功能测例本身不支持向串口发送信息,很难重构功能测例的代码实现向串口发送信息。更何况功能测例是测试均为最基本的指令,向串口发送数据你至少要保证基本的指令要实现正确、MMU要实现正确等等。

(2)监控程序支持向串口发送信息,但非常不完善,不仅是本身有bug,而且几乎不能向串口发送任何有用的错误信息,如果你的监控程序可以向串口正常发送信息了,那么你大概率监控程序已经调试通过了。

只有ucore能向串口发送出非常有用的信息,告诉读者哪一部分代码执行错误了,所以,读者看到这一章节时,已经基本到达了调试尾声了。

值得庆祝的是,这部分调试比之前两部分调试友善的多,因为ucore非常完善,可以修改ucore,通过ucore自带的cprintf语句轻松的向串口发送各种信息,事实上,进入这一部分,就与正常写软件调试相差无几了。

\section{具体实现}

根据前面的假定,认为读者本阶段在调试ucore操作系统,同时认为读者前面阶段已经进行了充足的测试,能够完美运行功能测例和监控程序。

在将ucore烧入Flash,将硬件烧入FPGA后,ucore操作系统会开始启动。

如果ucore没有正常启动,ucore会在串口输出的错误信息,同时进入内核调试程序,内核调试程序无太大用处。输出的错误信息中内含错误位置,定位到ucore操作系统对应的位置上,根据函数调用过程加入cprintf输出一些变量信息,cprintf用法和C语言printf用法无太大差异,详情可阅读操作系统文件。

重新编译ucore,重新烧入Flash,无需重新编译烧入FPGA,通过不断定位找出错误位置,思考硬件实现为什么错误。

在本阶段,编译ucore时间仅仅几秒钟,更可以通过cprintf自由的输出各种信息,相比之下调试可以算是非常友好。

最后祝读者调试顺利!

\chapter{外设连接}

CPU是计算机的核心所在,但是只有有了外设,才能称之为一台完整的计算机。在本节中,将介绍ROM、RAM、Flash、串口、VGA、七段数码管、LED灯、开关等外设的作用与设计(或使用方法)。

\section{ROM}

    \subsection{简介概述}
    ROM即为Read Only Memory,只读内存区,仅用于存储Boot Loader。Boot Loader是一段代码,CPU执行这段代码在系统启动初期将Flash中的uCore操作系统数据拷贝至RAM中。

    \subsection{接口定义}
        \longtablesixL{信号类型}{信号规格}{信号位宽}{信号名称}{来源/去向}{详细描述}
            input & wire & 1 & clk & thinpad\_top & 时钟信号\\
            input & wire & 1 & ce & thinpad\_top & 使能信号\\
            input & wire & 12 & addr & thinpad\_top & 读入地址\\
            \midrule
            output & reg & 32 & inst & thinpad\_top & 读出指令\\
        \longtableend

    \subsection{设计细节}
    Thinpad开发板并未集成ROM,ROM采用Verilog语言实现。由于Boot Loader代码较短,直接用case语句实现。在时钟的上升沿,若ROM被使能,给定一个地址,返回一条指令。由于ROM使用Verilog语言在FPGA中实现,故可以忽略读延迟,无需考虑时序关系。

    ROM采用字编址,字长32位。

\section{RAM}

    \subsection{简介概述}
    RAM即为Random Access Memory,随机访问存储器,断电数据消失,系统启动后作为系统的主存,load/store指令与此交互。

    \subsection{接口定义}
        \longtablesixL{信号类型}{信号规格}{信号位宽}{信号名称}{来源/去向}{详细描述}
            input & wire & 1 & ce\_n & thinpad\_top & 片选信号\\
            input & wire & 1 & oe\_n & thinpad\_top & 读使能信号\\
            input & wire & 1 & we\_n & thinpad\_top & 写使能地址\\
            input & wire & 4 & be\_n & thinpad\_top & 字节选择信号\\
            input & wire & 20 & addr & thinpad\_top & 读写地址\\
            \midrule
            inout & wire & 32 & data & thinpad\_top & 读写数据\\
        \longtableend

        注:n表示active low,低电平有效。

    \subsection{设计细节}
    RAM集成在Thinpad开发板上,共有2块,分别为Base RAM和Extern RAM,大小均为4MB,采用字编址,地址线20位,数据线32位。

    在使用RAM时,应阅读RAM的外设文档,理解清楚读写的时序关系,计算各个阶段的延迟。

    读RAM时,置片选信号为0,读使能信号为0,写使能信号为1,设置字节选择信号和地址,便可以读出数据。

    写RAM时,置片选信号为0,读使能信号为X(任意态),写使能信号为0,设置字节选择信号的地址,便可以写入数据。

    在实现的过程中,有一条极为重要的时序约束:写RAM时,字节选择信号和地址必须早于写使能信号到达,必须晚于写使能信号撤离,否则会造成覆盖写入(本希望写入1字节数据,写入多个字节)或是错误写入(本希望写入地址A,写入到地址B)。

    为了实现时序约束,在CPU内部采用状态机的方式实现,每条store指令都会被分为三个周期,第一个周期传递字节选择信号和地址,第二个周期拉低片选信号的写使能信号,第三个周期撤去字节选择信号和地址。

\section{Flash}

    \subsection{简介概述}
    Flash,快闪存储器,断电数据不消失,系统启动前存放uCore操作系统,仅在Boot阶段与此交互,从Flash中读出uCore。

    \subsection{接口定义}
        \longtablesixL{信号类型}{信号规格}{信号位宽}{信号名称}{来源/去向}{详细描述}
            input & wire & 1 & ce\_n & thinpad\_top & 片选信号\\
            input & wire & 1 & oe\_n & thinpad\_top & 读使能信号\\
            input & wire & 1 & we\_n & thinpad\_top & 写使能地址\\
            input & wire & 4 & byte\_n & thinpad\_top & 字节模式信号\\
            input & wire & 1 & rp\_n & thinpad\_top & 工作模式信号\\
            input & wire & 1 & vpen & thinpad\_top & 写保护\\
            input & wire & 23 & a & thinpad\_top & 读写地址\\
            \midrule
            inout & wire & 16 & data & thinpad\_top & 读写数据\\
        \longtableend

        注:n表示active low,低电平有效。

    \subsection{设计细节}
    Flash集成在Thinpad开发板上,大小为8MB,采用字节编址,地址线23位,数据线16位。

    在使用Flash时,应阅读Flash的外设文档,理解清楚读写的时序关系,计算各个阶段的延迟。

    读Flash时,置片选信号为0,读使能信号为0,写使能信号为1,字节模式信号为1(表示一次读出16个bit),工作模式信号为1(表示工作在正常模式),设置字节选择信号和地址,便可以读出数据。

    读Flash前,要先向Flash写入0xFF,将Flash转变为读模式。

\section{串口}

    \subsection{简介概述}
    串口,串行通信接口,用于计算机和外界进行通信。计算机内部均为并行信号,收发数据采用串行信号,串口进行并行数据与串行数据的转换,

    \subsection{接口定义}
        \longtablesixL{信号类型}{信号规格}{信号位宽}{信号名称}{来源/去向}{详细描述}
            input & wire & 8 & TxD\_data & thinpad\_top & 发数据\\
            input & wire & 1 & TxD\_busy & thinpad\_top & 发端口繁忙\\
            input & wire & 1 & TxD\_start & thinpad\_top & 发端口发送数据\\
            \midrule
            output & wire & 8 & RxD\_data & thinpad\_top & 收数据\\
            output & wire & 1 & RxD\_idle & thinpad\_top & 收端口空闲\\
            output & wire & 1 & RxD\_data\_ready & thinpad\_top & 收端口就绪\\

        \longtableend

    \subsection{设计细节}
    发送数据时,将TxD\_start置为1,TxD\_data传入数据。

    接收数据时,利用RxD\_data\_ready触发CPU中断,调用中断处理程序,中断读数据。

    <TODO>yyy

\section{VGA}

    \subsection{简介概述}
    VGA即为Video Graphics Array,视频图形阵列,用于向屏幕输出图像。

    \subsection{接口定义}
        \longtablesixL{信号类型}{信号规格}{信号位宽}{信号名称}{来源/去向}{详细描述}
            input & wire & 8 & pixel & thinpad\_top & 像素数据\\
            input & wire & 1 & hsync & thinpad\_top & 水平同步信号\\
            input & wire & 1 & vsync & thinpad\_top & 垂直同步信号\\
            input & wire & 1 & clk & thinpad\_top & 时钟信号\\
            input & wire & 1 & de & thinpad\_top & 数据使能信号\\
        \longtableend

    \subsection{设计细节}
    Thinpad开发板上提供了HDMI接口,可以向屏幕输出图像,其显示像素位宽为8,分别是R(3bit)、G(3bit)、B(2bit),最终实现了800x600刷新率75Hz的图像显示功能,

    VGA输出的原理为逐行扫描,同时需要多扫描一段区域并加入一些同步信号,工程模板中提供了逐行扫描模块的实现代码,详见VGA的外设文档。

    实现过程中需要一块显存。由于显示的原理为扫描,故需要一块区域存储向屏幕输出的数据,同时需要支持同时读写,给定一个地址读写一个8位宽的色彩像素。显存借助于Vivado提供的简单双端Block Memory IP核实现,读写数据宽度8位,读写深度480000(800x600)。

    在MMU中将0xBA000000-0xBA080000映射至IP核实现的显存位置,CPU向这段地址写入即访问显存,同时另一读端口接入逐行扫描的模块,即可输出图像。

\section{七段数码管}

    \subsection{简介概述}
    七段数码管用于显示十进制数字,用于开发前期的调试工作。

    \subsection{接口定义}
        \longtablesixL{信号类型}{信号规格}{信号位宽}{信号名称}{来源/去向}{详细描述}
            input & wire & 16 & leds & thinpad\_top & 数据\\
        \longtableend

    \subsection{设计细节}
    Thinpad开发板上集成了2个七段数码管,可以用来显示十进制数字,在调试初期非常有用,通过MMU将某一地址映射至七段数码管后可以用来显示指令内容、指令地址等等。

    七段数码管需要提供一个模块将位宽为4的二进制信号转换为0-F的七段数码管表示,在提供的模板工程里实现了这份代码,实现原理也非常简单,实现一个case语句即可。

    在功能测例里通过七段数码管中显示通过测例的数目,将功能测例里定义的七段数码管地址通过MMU实现映射,CPU将数据写至该地址即实现了七段数码管的输出,从而可以清楚地显示测试情况。

\section{LED灯}

    \subsection{简介概述}
    LED灯用于显示二进制数字,用于开发前期的调试工作。

    \subsection{接口定义}
        \longtablesixL{信号类型}{信号规格}{信号位宽}{信号名称}{来源/去向}{详细描述}
            input & wire & 16 & leds & thinpad\_top & 数据\\
        \longtableend
    \subsection{设计细节}
    Thinpad开发板上集成了16个LED灯,可以用来显示二进制数字,在调试初期非常有用,通过MMU将某一地址映射至LED灯后可以用来显示指令内容、指令地址等等。

    将leds信号相应位置1该灯即点亮,原理非常简单。

\section{开关}

    \subsection{简介概述}
    开关用于提供简单的控制功能,用于开发前期的调试工作。

    \subsection{接口定义}
        \longtablesixL{信号类型}{信号规格}{信号位宽}{信号名称}{来源/去向}{详细描述}
            output & wire & 32 & dip\_sw & thinpad\_top & 32个拨动开关\\
            output & wire & 6 & touch\_btn & thinpad\_top & 6个按动开关\\

        \longtableend
    \subsection{设计细节}
    Thinpad开发板上集成了32个拨动开关和6个按动开关(其中2个为clk和reset,已消除抖动,剩下4个需要自己实现消除抖动),用于开发前期提供最基础的控制功能。

    开发的整个过程中,都需要rst开关提供复位信号,清零CPU状态。

    开发初期为了Debug,可以七段数码管输出指令地址后8位,LED输出指令后16位,手按clk时钟进行调试。

    同时可以将左边32个波动开关设置为CPU控制信号,例如左边某个开关为1时,CPU进入调试模式,需要手按时钟。






% \section{算法实现}

% \subsubsection{BasicRNNCell}

%     \paragraph{算法原理} BasicRNNCell为最基础的一类Cell,从输入到输出,实现的仅为最简单的线性变换,并加以激活,无门机制控制记忆的写入与遗忘:
%     $$h_t = \tanh([h_{t-1}, x_t] \cdot W + b)$$
%     \image[4in]{BasicRNNCell}{BasicRNNCell}

%     \paragraph{实现方法} 实现时,只需模拟上述公式即可:

    % \begin{enumerate}
    %     \itembf{词向量导入}: 在\texttt{main.py}中导入预训练的词向量;
    %     \itembf{模型搭建}: 在\texttt{model.py}中实现\texttt{placeholder}等,搭建基于RNN的神经网络;
    %     \itembf{基本单元}: 在\texttt{cell.py}中实现BasicRNNCell, GRUCell, BasicLSTMCell等基础单元;
    %     \itembf{模型可视化}: 在\texttt{main.py}中加入TensorBoard可视化代码。
    % \end{enumerate}

    % \begin{lstlisting}
    % for vocab in vocab_list:
    %     if vocab in embed_dict:
    %         embed.append(embed_dict[vocab])
    %     else:
    %         embed.append([0.0] * FLAGS.embed_units)
    % \end{lstlisting}

    % \tablethreeL{变量}{形状}{含义}
    %     $x_t$ & $[batch\_size \times embed\_units]$ & 当前时刻的输入 \\
    %     \midrule
    %     $z_t$ & $[batch\_size \times num\_units]$ & update门,候选状态对新状态的影响 \\
    %     $r_t$ & $[batch\_size \times num\_units]$ & reset门,旧状态对候选状态的影响 \\
    %     \midrule
    %     $W_z, b_z$ & $[(embed\_units + num\_units) \times num\_units]$, $[num\_units]$ & update门的变换矩阵、偏置 \\
    %     $W_r, b_r$ & $[(embed\_units + num\_units) \times num\_units]$, $[num\_units]$ & reset门的变换矩阵、偏置 \\
    %     $W, b$ & $[(embed\_units + num\_units) \times num\_units]$, $[num\_units]$ & 从旧状态到候选状态的变换矩阵、偏置 \\
    %     \midrule
    %     $\tilde{h_t}$ & $[batch\_size \times num\_units]$ & 候选状态 \\
    %     $h_t$ & $[batch\_size \times num\_units]$ & 产生的新状态 \\
    % \tableend

    % \image[6in]{loss_gru}{GRUCell loss-epoch曲线}

% \begin{thebibliography}{9}
%     \bibitem{GRUPaper} http://arxiv.org/abs/1406.1078
%     \bibitem{LSTMPaper} http://arxiv.org/abs/1409.2329
%     \bibitem{RNNTutorial} https://zhuanlan.zhihu.com/p/28196873
%     \bibitem{LSTMTutorial} http://colah.github.io/posts/2015-08-Understanding-LSTMs/
%     \bibitem{GRUTutorial} http://blog.csdn.net/meanme/article/details/48845793
%     \bibitem{BiLSTMTutorial} http://blog.csdn.net/wuzqChom/article/details/75453327
% \end{thebibliography}

\end{document}
