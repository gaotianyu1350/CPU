\chapter{外设连接}

CPU是计算机的核心所在,但是只有有了外设,才能称之为一台完整的计算机。在本节中,将介绍ROM、RAM、Flash、串口、VGA、七段数码管、LED灯、开关等外设的作用与设计(或使用方法)。

\section{ROM}

    \subsection{简介概述}
    ROM即为Read Only Memory,只读内存区,仅用于存储Boot Loader。Boot Loader是一段代码,CPU执行这段代码在系统启动初期将Flash中的uCore操作系统数据拷贝至RAM中。

    \subsection{接口定义}
        \longtablesixL{信号类型}{信号规格}{信号位宽}{信号名称}{来源/去向}{详细描述}
            input & wire & 1 & clk & thinpad\_top & 时钟信号\\
            input & wire & 1 & ce & thinpad\_top & 使能信号\\
            input & wire & 12 & addr & thinpad\_top & 读入地址\\
            \midrule
            output & reg & 32 & inst & thinpad\_top & 读出指令\\
        \longtableend

    \subsection{设计细节}
    Thinpad开发板并未集成ROM,ROM采用Verilog语言实现。由于Boot Loader代码较短,直接用case语句实现。在时钟的上升沿,若ROM被使能,给定一个地址,返回一条指令。由于ROM使用Verilog语言在FPGA中实现,故可以忽略读延迟,无需考虑时序关系。

    ROM采用字编址,字长32位。

\section{RAM}

    \subsection{简介概述}
    RAM即为Random Access Memory,随机访问存储器,断电数据消失,系统启动后作为系统的主存,load/store指令与此交互。

    \subsection{接口定义}
        \longtablesixL{信号类型}{信号规格}{信号位宽}{信号名称}{来源/去向}{详细描述}
            input & wire & 1 & ce\_n & thinpad\_top & 片选信号\\
            input & wire & 1 & oe\_n & thinpad\_top & 读使能信号\\
            input & wire & 1 & we\_n & thinpad\_top & 写使能地址\\
            input & wire & 4 & be\_n & thinpad\_top & 字节选择信号\\
            input & wire & 20 & addr & thinpad\_top & 读写地址\\
            \midrule
            inout & wire & 32 & data & thinpad\_top & 读写数据\\
        \longtableend

        注:n表示active low,低电平有效。

    \subsection{设计细节}
    RAM集成在Thinpad开发板上,共有2块,分别为Base RAM和Extern RAM,大小均为4MB,采用字编址,地址线20位,数据线32位。

    在使用RAM时,应阅读RAM的外设文档,理解清楚读写的时序关系,计算各个阶段的延迟。

    读RAM时,置片选信号为0,读使能信号为0,写使能信号为1,设置字节选择信号和地址,便可以读出数据。

    写RAM时,置片选信号为0,读使能信号为X(任意态),写使能信号为0,设置字节选择信号的地址,便可以写入数据。

    在实现的过程中,有一条极为重要的时序约束:写RAM时,字节选择信号和地址必须早于写使能信号到达,必须晚于写使能信号撤离,否则会造成覆盖写入(本希望写入1字节数据,写入多个字节)或是错误写入(本希望写入地址A,写入到地址B)。

    为了实现时序约束,在CPU内部采用状态机的方式实现,每条store指令都会被分为三个周期,第一个周期传递字节选择信号和地址,第二个周期拉低片选信号的写使能信号,第三个周期撤去字节选择信号和地址。

\section{Flash}

    \subsection{简介概述}
    Flash,快闪存储器,断电数据不消失,系统启动前存放uCore操作系统,仅在Boot阶段与此交互,从Flash中读出uCore。

    \subsection{接口定义}
        \longtablesixL{信号类型}{信号规格}{信号位宽}{信号名称}{来源/去向}{详细描述}
            input & wire & 1 & ce\_n & thinpad\_top & 片选信号\\
            input & wire & 1 & oe\_n & thinpad\_top & 读使能信号\\
            input & wire & 1 & we\_n & thinpad\_top & 写使能地址\\
            input & wire & 4 & byte\_n & thinpad\_top & 字节模式信号\\
            input & wire & 1 & rp\_n & thinpad\_top & 工作模式信号\\
            input & wire & 1 & vpen & thinpad\_top & 写保护\\
            input & wire & 23 & a & thinpad\_top & 读写地址\\
            \midrule
            inout & wire & 16 & data & thinpad\_top & 读写数据\\
        \longtableend

        注:n表示active low,低电平有效。

    \subsection{设计细节}
    Flash集成在Thinpad开发板上,大小为8MB,采用字节编址,地址线23位,数据线16位。

    在使用Flash时,应阅读Flash的外设文档,理解清楚读写的时序关系,计算各个阶段的延迟。

    读Flash时,置片选信号为0,读使能信号为0,写使能信号为1,字节模式信号为1(表示一次读出16个bit),工作模式信号为1(表示工作在正常模式),设置字节选择信号和地址,便可以读出数据。

    读Flash前,要先向Flash写入0xFF,将Flash转变为读模式。

\section{串口}

    \subsection{简介概述}
    串口,串行通信接口,用于计算机和外界进行通信。计算机内部均为并行信号,收发数据采用串行信号,串口进行并行数据与串行数据的转换,

    \subsection{接口定义}
        \longtablesixL{信号类型}{信号规格}{信号位宽}{信号名称}{来源/去向}{详细描述}
            input & wire & 8 & TxD\_data & thinpad\_top & 发数据\\
            input & wire & 1 & TxD\_busy & thinpad\_top & 发端口繁忙\\
            input & wire & 1 & TxD\_start & thinpad\_top & 发端口发送数据\\
            \midrule
            output & wire & 8 & RxD\_data & thinpad\_top & 收数据\\
            output & wire & 1 & RxD\_idle & thinpad\_top & 收端口空闲\\
            output & wire & 1 & RxD\_data\_ready & thinpad\_top & 收端口就绪\\

        \longtableend

    \subsection{设计细节}
    发送数据时,将TxD\_start置为1,TxD\_data传入数据。

    接收数据时,利用RxD\_data\_ready触发CPU中断,调用中断处理程序,中断读数据。

    <TODO>yyy

\section{VGA}

    \subsection{简介概述}
    VGA即为Video Graphics Array,视频图形阵列,用于向屏幕输出图像。

    \subsection{接口定义}
        \longtablesixL{信号类型}{信号规格}{信号位宽}{信号名称}{来源/去向}{详细描述}
            input & wire & 8 & pixel & thinpad\_top & 像素数据\\
            input & wire & 1 & hsync & thinpad\_top & 水平同步信号\\
            input & wire & 1 & vsync & thinpad\_top & 垂直同步信号\\
            input & wire & 1 & clk & thinpad\_top & 时钟信号\\
            input & wire & 1 & de & thinpad\_top & 数据使能信号\\
        \longtableend

    \subsection{设计细节}
    Thinpad开发板上提供了HDMI接口,可以向屏幕输出图像,其显示像素位宽为8,分别是R(3bit)、G(3bit)、B(2bit),最终实现了800x600刷新率75Hz的图像显示功能,

    VGA输出的原理为逐行扫描,同时需要多扫描一段区域并加入一些同步信号,工程模板中提供了逐行扫描模块的实现代码,详见VGA的外设文档。

    实现过程中需要一块显存。由于显示的原理为扫描,故需要一块区域存储向屏幕输出的数据,同时需要支持同时读写,给定一个地址读写一个8位宽的色彩像素。显存借助于Vivado提供的简单双端Block Memory IP核实现,读写数据宽度8位,读写深度480000(800x600)。

    在MMU中将0xBA000000-0xBA080000映射至IP核实现的显存位置,CPU向这段地址写入即访问显存,同时另一读端口接入逐行扫描的模块,即可输出图像。

\section{七段数码管}

    \subsection{简介概述}
    七段数码管用于显示十进制数字,用于开发前期的调试工作。

    \subsection{接口定义}
        \longtablesixL{信号类型}{信号规格}{信号位宽}{信号名称}{来源/去向}{详细描述}
            input & wire & 16 & leds & thinpad\_top & 数据\\
        \longtableend

    \subsection{设计细节}
    Thinpad开发板上集成了2个七段数码管,可以用来显示十进制数字,在调试初期非常有用,通过MMU将某一地址映射至七段数码管后可以用来显示指令内容、指令地址等等。

    七段数码管需要提供一个模块将位宽为4的二进制信号转换为0-F的七段数码管表示,在提供的模板工程里实现了这份代码,实现原理也非常简单,实现一个case语句即可。

    在功能测例里通过七段数码管中显示通过测例的数目,将功能测例里定义的七段数码管地址通过MMU实现映射,CPU将数据写至该地址即实现了七段数码管的输出,从而可以清楚地显示测试情况。

\section{LED灯}

    \subsection{简介概述}
    LED灯用于显示二进制数字,用于开发前期的调试工作。

    \subsection{接口定义}
        \longtablesixL{信号类型}{信号规格}{信号位宽}{信号名称}{来源/去向}{详细描述}
            input & wire & 16 & leds & thinpad\_top & 数据\\
        \longtableend
    \subsection{设计细节}
    Thinpad开发板上集成了16个LED灯,可以用来显示二进制数字,在调试初期非常有用,通过MMU将某一地址映射至LED灯后可以用来显示指令内容、指令地址等等。

    将leds信号相应位置1该灯即点亮,原理非常简单。

\section{开关}

    \subsection{简介概述}
    开关用于提供简单的控制功能,用于开发前期的调试工作。

    \subsection{接口定义}
        \longtablesixL{信号类型}{信号规格}{信号位宽}{信号名称}{来源/去向}{详细描述}
            output & wire & 32 & dip\_sw & thinpad\_top & 32个拨动开关\\
            output & wire & 6 & touch\_btn & thinpad\_top & 6个按动开关\\

        \longtableend
    \subsection{设计细节}
    Thinpad开发板上集成了32个拨动开关和6个按动开关(其中2个为clk和reset,已消除抖动,剩下4个需要自己实现消除抖动),用于开发前期提供最基础的控制功能。

    开发的整个过程中,都需要rst开关提供复位信号,清零CPU状态。

    开发初期为了Debug,可以七段数码管输出指令地址后8位,LED输出指令后16位,手按clk时钟进行调试。

    同时可以将左边32个波动开关设置为CPU控制信号,例如左边某个开关为1时,CPU进入调试模式,需要手按时钟。



