\chapter{引言}

\section{背景}

本项目的顶级需求为在Thinpad开发板上运行ucore操作系统。其衍生需求为设计一个基于MIPS 32架构的CPU,以实现ucore所需的46条指令、精确异常与外设调度。

本项目是计算机组成原理和软件工程课程的联合实验,项目需求方为计算机组成原理与软件工程课程。

本项目的承担方为NonExist小组,包括计55班张钰晖、计55班杨一滨、计54班周正平3位成员。

\section{编写目的}

本需求文档的编写目的如下:

\begin{enumerate}
    \item {\bf 需求分析:} 明确软件(ucore)对硬件(CPU)的具体需求,及需完成的功能
    \item {\bf 系统概述:} 对系统总体框架进行清晰完整的描述,确定迭代开发计划
    \item {\bf 学习目标:} 明确开发所需技术,订立学习目标
\end{enumerate}

其目标读者包括NonExist小组的开发者、成品CPU的使用者、项目的评估者。

\section{项目概览}

首先给出本项目的需求分析图:

<TODO>

\section{定义}

<TODO>

\section{参考资料}

<TODO>
