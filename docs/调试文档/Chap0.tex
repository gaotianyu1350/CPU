\chapter{文档说明}

写码一小时,调试一整天。在真正的整个开发过程中,开发、测试、调试的时间几乎相等,实现CPU的过程中Bug是不可避免的,硬件调试也是非常痛苦的,不像软件开发可以轻易的输出中间结果,硬件出了问题没有任何方法可以轻松的print到屏幕上,因此调试非常重要。

本文档是NonExist组软工计原联合项目的调试文档,本文档主要描述了在实现CPU的过程中我们如何进行调试。

调试文档将项目分成了以下部分:

    \begin{enumerate}
        \itembf{仿真调试}:分阶段用Verilog实现SOPC进行软件仿真调试。
        \itembf{硬件调试}:连接在板子上手按时钟进行调试。
        \itembf{输出调试}:用串口输出信息进行调试。
    \end{enumerate}


调试文档每个章节遵从以下介绍流程:

    \begin{enumerate}
        \itembf{简介概述}:简要介绍调试的方法。
        \itembf{具体实现}:详细介绍调试的方法和注意事项。
    \end{enumerate}

希望本文档能给读者带来裨益。
