\chapter{输出调试}

\section{简介概述}

如果读者已经把串口调试通过,可以通过串口收发信息,那么可以通过修改软件直接将数据发送至串口查看硬件信息。

听起来非常棒,就像是调试软件中可以向屏幕print信息一样,但事实上,现实并没有那么美好。

(1)功能测例本身不支持向串口发送信息,很难重构功能测例的代码实现向串口发送信息。更何况功能测例是测试均为最基本的指令,向串口发送数据你至少要保证基本的指令要实现正确、MMU要实现正确等等。

(2)监控程序支持向串口发送信息,但非常不完善,不仅是本身有bug,而且几乎不能向串口发送任何有用的错误信息,如果你的监控程序可以向串口正常发送信息了,那么你大概率监控程序已经调试通过了。

只有ucore能向串口发送出非常有用的信息,告诉读者哪一部分代码执行错误了,所以,读者看到这一章节时,已经基本到达了调试尾声了。

值得庆祝的是,这部分调试比之前两部分调试友善的多,因为ucore非常完善,可以修改ucore,通过ucore自带的cprintf语句轻松的向串口发送各种信息,事实上,进入这一部分,就与正常写软件调试相差无几了。

\section{具体实现}

根据前面的假定,认为读者本阶段在调试ucore操作系统,同时认为读者前面阶段已经进行了充足的测试,能够完美运行功能测例和监控程序。

在将ucore烧入Flash,将硬件烧入FPGA后,ucore操作系统会开始启动。

如果ucore没有正常启动,ucore会在串口输出的错误信息,同时进入内核调试程序,内核调试程序无太大用处。输出的错误信息中内含错误位置,定位到ucore操作系统对应的位置上,根据函数调用过程加入cprintf输出一些变量信息,cprintf用法和C语言printf用法无太大差异,详情可阅读操作系统文件。

重新编译ucore,重新烧入Flash,无需重新编译烧入FPGA,通过不断定位找出错误位置,思考硬件实现为什么错误。

在本阶段,编译ucore时间仅仅几秒钟,更可以通过cprintf自由的输出各种信息,相比之下调试可以算是非常友好。

最后祝读者调试顺利!