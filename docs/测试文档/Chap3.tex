\chapter{ucore操作系统}

\section{简介概述}

由于前面3个阶段已经对硬件进行了充分而严谨的测试,我们在ucore调试阶段仅遇到了2-3个bug,耗时2天。

ucore操作系统\footnote{https://github.com/chyh1990/ucore-thumips}是我们需要完成的终极系统测试。其内核使用C语言与MIPS 32汇编语言编写,并提供若干使用C语言编写的用户态应用程序。
其编译、仿真运行方式在需求文档中有详细的介绍,此处不再赘述。

ucore是一个完整的操作系统,它综合使用了CPU实现的全部46条指令,是一个完整、全面的系统测试。

相比32位监控程序,ucore增加了对MMU(与其中的TLB机制)的需求,其访存体系更加复杂;此外还增加了对TLB异常的处理,其处理方式不同于通用异常,因而对CPU的异常处理需求有所增强。

ucore的主要流程如下:

\begin{enumerate}
    \item {\bf Boot阶段}:硬件初始化时将CP0的Status、Random寄存器初始化,并将TLB的``隐藏''位设为禁止匹配,最后置PC值为VA 0xBFC00000。
    此后读取ROM中存储的BootLoader,将Flash中存储的监控程序拷贝至RAM。拷贝结束后跳转至操作系统入口VA 0x80000000。
    \item {\bf 系统初始化}:这一阶段无效化所有TLB表项、打开时钟与串口中断、输出内核调试信息。此后进行内存管理、进程调度等高层初始化。
    最终应输出信息``user sh is running!!!'',中途不能发生系统panic。
    \item {\bf 启动完毕}:系统初始化完毕后进入用户态,可以运行如下列表中的应用程序以验证操作系统已正常运行:

    \tabletwoL{应用程序}{功能(期望输出)}
        pwd & 打印当前完整路径 \\
        cat & 打印一个文件的内容 \\
        sh & 打印一条语句``user sh is running!!!'' \\
        forktest & 连续调用fork()函数,新建一系列线程,并打印信息 \\
        yield & 连续调用yield()函数,要求进程重新调度,并打印信息 \\
        hello & 打印一条语句``Hello world!!.''并显示当前进程编号 \\
        faultreadkernel & 在用户态访问内核态地址,结果是系统panic \\
        faultread & 在用户态访问非法内存地址0x0,结果是系统panic \\
        badarg & 先调用fork()再调用yield(),打印进程调度信息 \\
        pgdir & 打印页表信息 \\
        exit & 先调用fork()再连续调用yield(),打印进程调度与退出信息 \\
        sleep & 每隔一定时间打印一条信息 \\
    \tableend
\end{enumerate}

\section{测试范畴}

\subsection{测试覆盖面}

\image[5in]{ucore}{ucore操作系统测试范畴(绿-可以测试;灰-不能测试)}

ucore操作系统提供对CPU的完整测试。它测试了CPU的全部部件的正确性:

\begin{enumerate}
    \item {\bf 五级流水线}:包括IF、ID、EX、MEM、WB共5个模块。
    \item {\bf 寄存器}:包括寄存器堆(32个通用寄存器)、HI/LO寄存器共2个模块。
    \item {\bf CP0}:包括CP0共1个模块。
    \item {\bf Control}:包括Control(流水线控制器)共1个模块。
    \item {\bf MMU}:包括TLB共1个模块。
    \item {\bf ROM}:包括ROM(存储BootLoader)共1个外部设备。
    \item {\bf RAM}:包括RAM(用作内存)共1个外部设备。
    \item {\bf Flash}:包括Flash(存储监控程序)共1个外部设备。
    \item {\bf 串口}:包括串口(用于与用户终端交互)共1个外部设备。
\end{enumerate}

\subsection{测试要点}

除去功能测例、32位监控程序已经覆盖的测试要点之外,ucore的测试要点主要集中于访存的TLB机制与TLB MISS异常的处理方面:

\begin{enumerate}
    \item {\bf 访存TLB机制}:ucore需要TLB对kuseg、kseg2段的访存进行地址映射,要求支持以TLBWI、TLBWR指令进行的写入操作。
    \item {\bf TLB MISS异常}:ucore对TLB MISS异常进行处理,这需要硬件正确实现对Index、EntryLo、EntryHi等寄存器的维护。
    \item {\bf 其他异常}:包括地址未对齐、算术溢出等异常。尽管在需求文档中已经分析,此类异常无需实现(因为即使实现了,在发生此类异常时ucore依旧会陷入panic),
    但在调试时可能会由于其他的硬件错误触发此类异常。因而CPU需增加对此类异常的支持,以方便调试。
\end{enumerate}

\section{测试方法}

ucore的系统测试步骤类似32位监控程序,如下:

\begin{enumerate}
    \item {\bf 系统初始化}

    \begin{enumerate}
        \item 将BootLoader编译后写入ROM对应的Verilog文件中。
        \item 将ucore编译后烧写入Flash的地址0x0处。
        \item 将CPU的.bit设计文件烧写入开发板FPGA中。
        \item 点击开发板上Reset开关启动系统。
    \end{enumerate}

    \item {\bf 启动客户端(监控程序终端)}

    \begin{enumerate}
        \item 将开发板的USB接口连接至PC端,以建立串口通信。
        \item 在串口调试精灵界面观察串口输出。ucore应依次输出如下内容:

        \begin{enumerate}
            \item 内核初始化信息,包括内存管理初始化、文件系统初始化等
            \item 初始化成功信息``user sh is running!!!''
        \end{enumerate}

        \item 向串口发送pwd、cat、ls等命令执行用户态程序,并与ucore在qemu模拟器上执行相同命令时的输出比对,确认ucore已正确运行。
    \end{enumerate}

\end{enumerate}

\section{测试结果}

经过前面几个阶段的铺垫,我们很快便完成了ucore的调试。经测试,ucore操作系统可以正常启动,并且运行各个用户态程序可以观测到预期的输出:

<TODO>:截图

\section{问题与解决}

<TODO>:建议简要描述一下这一阶段的几个bug

\begin{enumerate}
    \item {\bf TLBWI、TLBWR指令无效bug}:\url{http://47.94.142.165:8088/gitlab/PRJ11_NonExist/CPU/commit/66b30f55cb6210dac9e78d75e9d900bfc08d1647}
    \item {\bf TLB MISS异常增加}:\url{http://47.94.142.165:8088/gitlab/PRJ11_NonExist/CPU/commit/e05335f0794c049d2729e8e803fa18676a2b6c99}
    \item {\bf 应使用Ebase存储异常处理入口地址}:\url{http://47.94.142.165:8088/gitlab/PRJ11_NonExist/CPU/commit/73adcbb64baeff0d5ba79bc5a18e0724c8cf526a}
\end{enumerate}
