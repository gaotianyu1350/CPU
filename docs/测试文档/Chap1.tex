\chapter{功能测例}

\section{简介概述}

功能测例由汇编语言实现,主要用于测试CPU指令实现是否正确。

功能测例涵盖了91项测试,其中根据我们CPU最终完成情况,75项是可测测例。

在此基础上,我们又增添了针对TLB操作指令TLBWI和TLBWR两条指令的测例,故总计77项测例。

<TODO>读一读功能测例的程序,贴一下代码,分析一下原理,说一说测试程序为什么很好。(极大地减轻了我们写功能测例的时间,对比之前文档可以看出来学长都是自己搞的)

\section{测试范畴}

功能测例主要测试了CPU以下部件是否实现正确。

    \begin{enumerate}
        \itembf{指令流水}
        \itembf{<TODO>抄一抄PPT}
    \end{enumerate}

\section{测试方式}

\subsection{仿真阶段}

在仿真阶段,通过我们自己实现的RAM模块,将功能测例编译后解析成文件,在RAM的initial语句导入该文件,将CPU指令计数器置为0x80000000,开始仿真运行。

通过阅读功能测例的代码我们可知,19号寄存器的数值存放了功能测例的通过条数。

\subsection{硬件阶段}

在硬件阶段,将功能测例定义的七段数码管地址通过MMU映射至开发板的七段数码管,将编译后的功能测例烧入BaseRAM中,则七段数码管显示的即为通过条数。

\section{测试结果}

\subsection{仿真阶段}

77条功能测例全部通过,见以下表格。

\subsection{硬件阶段}

时钟频率25MHz下,77条功能测例全部通过,见以下表格。

<TODO>此处抄CPU工程readme表格。

\section{问题}

功能测例自己会破坏自己,不能一次测太多。