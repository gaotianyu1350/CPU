\chapter{功能测例}

\section{简介概述}

功能测例由汇编语言实现,主要用于测试CPU指令实现是否正确。

功能测例涵盖了91项测试,其中根据我们CPU最终完成情况,75项是可测测例。

在此基础上,我们又增添了针对TLB操作指令TLBWI和TLBWR两条指令的测例,故总计77项测例。

% <TODO>读一读功能测例的程序,贴一下代码,分析一下原理,说一说测试程序为什么很好。(极大地减轻了我们写功能测例的时间,对比之前文档可以看出来学长都是自己搞的)

功能测例由MIPS 32汇编语言编写。以下以第1个测试点ADD的流程为例,说明其程序结构。

\image[3.5in]{test_frame}{功能测例程序框架}

\begin{enumerate}
    \item {\bf 主程序}:整个测试程序连续执行若干\emph{测试点},每个测试点结束后停顿1s。

    例如,以下为主程序start.S中调用测试点ADD的代码片段:

    \begin{lstlisting}
    inst_test:
        jal n1_add_test   #add
        nop
        jal wait_1s
        nop
    \end{lstlisting}

    \item {\bf 测试点}:每个测试点对应于一条指令。

    例如,n1\_add\_test 测试点的代码片段如下:

    \begin{lstlisting}[emph={TEST\_ADD}, caption={inst/n1\_add.S}]
    9   LEAF(n1_add_test)
	10      lui a0, 0x100
	11      li v0, 0x0
	12  ###test inst
	13    	TEST_ADD(0x0480ff04, 0x40933204, 0x45143108)    # num1, num2 both random
	14      TEST_ADD(0x2a19dd40, 0xa87971e0, 0xd2934f20)
    ...
	211     TEST_ADD(0x25e5fad8, 0x00000000, 0x25e5fad8)    # num1 == 0 or num2 == 0
	212     TEST_ADD(0x00000000, 0xdcaf5e62, 0xdcaf5e62)
    ...
	262     TEST_ADD(0x00000000, 0x00000000, 0x00000000)    # num1 == 0 and num2 == 0
	263 ###detect exception
	264     bne v0, zero, inst_error
	265     nop
	266 ###score ++
	267     addiu s3, s3, 1
	268 ###output a0|s3
	269 inst_error:
	270     or t0, a0, s3
	271     sw t0, 0(s1)
	272     jr ra
	273     nop
	274 END(n1_add_test)
    \end{lstlisting}

    \paragraph{初始化}
    其中,第 10 行 a0 寄存器存储了测试功能点编号值,这里为 0x1;第 11 行 v0 寄存器存储的是例外检
    测,初始值为 0,如果程序出现例外会置为 0xFFFF0000。

    \paragraph{指令执行}
    第 12 行到第 262 行为测试指令,分别测试了加数 1 和 2 都是随机数、加数 1 或 2 是 0、加数 1 和 2 都是 0 的情况。
    其中,宏TEST\_ADD 调用宏ADD执行指令,并判断目的寄存器 s0 的值是否与正确结果 s2 寄存器的值一致;
    宏ADD 将加数 1 装载于寄存器 t0,加数 2 装载于寄存器 t1,加法结果位于寄存器 s0:

    \begin{figure}[H]
        \centering
        \begin{minipage}{0.4\textwidth}
            \begin{lstlisting}[emph={ADD}, language=c]
            #define TEST_ADD(vs, vt, vd) \
                ADD(vs, vt); \
                li s2, vd; \
                bne s0, s2, inst_error; \
                nop
            \end{lstlisting}
        \end{minipage}
        \hspace{0.5cm}
        \begin{minipage}{0.4\textwidth}
            \begin{lstlisting}[language=c]
            // ADD
            #define ADD(v0, v1) \
                li t0, v0; \
                li t1, v1; \
                add s0, t0, t1
            \end{lstlisting}
        \end{minipage}
    \end{figure}

    \paragraph{错误检测}
    第 263-265 行通过寄存器v0判断测试过程中是否出现例外,
    正确情况下没有例外;如出现例外则跳转到inst\_error。

    \paragraph{测试通过时计数器+1}
    第 266-267 行,s3 寄存器存测试分值,测试通过一个功能点加 1 分;第 268-273 行
    为测试结束,将测试编号和测试分值输出显示并返回主程序。

\end{enumerate}

可以看出,功能测例对指令集有着较高的覆盖率、较全面鲁棒的测试数据、较清晰的代码结构,并且可以自动验证执行的正确性,帮助开发者大量减少了自行编写测例的时间。

\section{测试范畴}

\subsection{测试覆盖面}

\image[5in]{functest}{功能测例测试范畴(绿-可以测试;灰-不能测试)}

功能测例主要测试了CPU以下部件是否实现正确:

\begin{enumerate}
    \item {\bf 五级流水线}:包括IF、ID、EX、MEM、WB共5个模块。
    \item {\bf 寄存器}:包括寄存器堆(32个通用寄存器)、HI/LO寄存器共2个模块。
    \item {\bf Control}:包括Control(流水线控制器)共1个模块。
    \item {\bf RAM}:包括RAM(用作内存)共1个外部设备。
\end{enumerate}

% TODO:功能测例不是有测异常的代码吗?
功能测例不能全面测试的CPU部件包括:

\begin{enumerate}
    \item {\bf CP0}:功能测例不能测试TLB MISS、中断等异常,故不能全面测试CP0。
    \item {\bf MMU}:功能测例不需要TLB进行地址映射,故不能全面测试MMU。
    \item {\bf ROM}:功能测例不需要BootLoader进行引导,故不能测试ROM。
    \item {\bf Flash}:功能测例烧录在BaseRAM中,不需要存入硬盘,故不能测试Flash。
    \item {\bf 串口}:功能测例不需要标准输入/输出,故不能测试串口。
\end{enumerate}

\subsection{测试要点}

功能测例的测试要点包括如下几个部分:

\begin{enumerate}
    \item {\bf 算术运算指令}:共12条,测试加、减、乘等指令。
    \item {\bf 逻辑运算指令}:共14条,测试与、或、移位等指令。
    \item {\bf 分支跳转指令}:共10条,测试B/J型指令。
    \item {\bf HI/LO寄存器指令}:共4条,测试对HI/LO寄存器的读写。
    \item {\bf CP0相关指令}:共4条,包括SYSCALL、ERET、读/写CP0等指令。这一部分也测试部分异常处理行为。
    \item {\bf 访存指令}:共8条,包括L/S型指令。
    \item {\bf 异常指令}:共13条,其中的指令会触发整型溢出异常、地址未对齐异常、发生于延迟槽的异常等。
    \item {\bf 延迟槽指令}:共10条,其中的被测指令位于延迟槽中。
\end{enumerate}

此外,笔者新增了2条针对TLB的功能测例:

\begin{enumerate}
    \item {\bf TLBWI}:测试指令TLBWI。测试文件如下:

    \begin{lstlisting}[caption={n92\_tlbwi.S}, emph={TEST\_TLBWI}]
    LEAF(n92_tlbwi_test)
        .set noreorder
        lui  a0, 0x100
        li  v0, 0x0
    ###test inst
        TEST_TLBWI(0x0, 0x4002, 0x4042, 0x0, 0x61754443, 0x000007a0)
        TEST_TLBWI(0x0, 0x4002, 0x4042, 0x1, 0x5c4fb45a, 0x00000aac)
        TEST_TLBWI(0x0, 0x4002, 0x4042, 0x2, 0x14908300, 0x00000ae8)
        TEST_TLBWI(0x0, 0x4002, 0x4042, 0x3, 0x516da739, 0x000000cc)
        TEST_TLBWI(0x0, 0x4002, 0x4042, 0x4, 0x85675a34, 0x00000510)
        TEST_TLBWI(0x0, 0x4002, 0x4042, 0x5, 0x0e4dac98, 0x00000040)
        TEST_TLBWI(0x0, 0x4002, 0x4042, 0x6, 0xd9c6eddb, 0x00000180)
        TEST_TLBWI(0x0, 0x4002, 0x4042, 0x7, 0x5753dd01, 0x00000ca0)
        TEST_TLBWI(0x0, 0x4002, 0x4042, 0x8, 0xe543b9f3, 0x0000031c)
        TEST_TLBWI(0x0, 0x4002, 0x4042, 0x9, 0x4726aca2, 0x00000cf8)
        TEST_TLBWI(0x0, 0x4002, 0x4042, 0xa, 0xb022040a, 0x00000800)
        TEST_TLBWI(0x0, 0x4002, 0x4042, 0xb, 0x5ca0fd00, 0x00000834)
        TEST_TLBWI(0x0, 0x4002, 0x4042, 0xc, 0x063ba000, 0x00000c64)
        TEST_TLBWI(0x0, 0x4002, 0x4042, 0xd, 0xc2268cfe, 0x000001e8)
        TEST_TLBWI(0x0, 0x4002, 0x4042, 0xe, 0x1611444c, 0x00000484)
        TEST_TLBWI(0x0, 0x4002, 0x4042, 0xf, 0x33cc6f2a, 0x000001dc)
    ###detect exception
        bne v0, zero, inst_error
        nop
    ###score ++
        addiu s3, s3, 1
    ###output a0|s3
    inst_error:
        or t0, a0, s3
        sw t0, 0(s1)
        jr ra
        nop
    END(n92_tlbwi_test)
    \end{lstlisting}

    宏TEST\_TLBWI会依次将CP0的Index寄存器赋值为0x0, 0x1, ..., 0xF,并依次调用TLBWI指令写TLB表项。最终,通过先向基址处写入数据再读出,判断其是否与原数据相等,作为测试通过的依据:

    \begin{lstlisting}[caption={inst\_test.h}]
    /* 92
     * TEST_TLBWI(ENTRYHI, ENTRYLO0, ENTRYLO1, INDEX, data, base_addr)
     */
    #define TEST_TLBWI(entryhi, entrylo0, entrylo1, index, data, base_addr) \
    	li s2, 0; \
    	mtc0 s2, c0_entrylo0; \
    	mtc0 s2, c0_entrylo1; \
    	mtc0 s2, c0_entryhi; \
    	li s2, 0; \
    	mtc0 s2, c0_index; \
    	tlbwi; \
    	li s2, 1; \
    	mtc0 s2, c0_index; \
    	tlbwi; \
    	li s2, 2; \
    	mtc0 s2, c0_index; \
    	tlbwi; \
    	li s2, 3; \
    	mtc0 s2, c0_index; \
    	tlbwi; \
    	li s2, 4; \
    	mtc0 s2, c0_index; \
    	tlbwi; \
    	li s2, 5; \
    	mtc0 s2, c0_index; \
    	tlbwi; \
    	li s2, 6; \
    	mtc0 s2, c0_index; \
    	tlbwi; \
    	li s2, 7; \
    	mtc0 s2, c0_index; \
    	tlbwi; \
    	li s2, 8; \
    	mtc0 s2, c0_index; \
    	tlbwi; \
    	li s2, 9; \
    	mtc0 s2, c0_index; \
    	tlbwi; \
    	li s2, 10; \
    	mtc0 s2, c0_index; \
    	tlbwi; \
    	li s2, 11; \
    	mtc0 s2, c0_index; \
    	tlbwi; \
    	li s2, 12; \
    	mtc0 s2, c0_index; \
    	tlbwi; \
    	li s2, 13; \
    	mtc0 s2, c0_index; \
    	tlbwi; \
    	li s2, 14; \
    	mtc0 s2, c0_index; \
    	tlbwi; \
    	li s2, 15; \
    	mtc0 s2, c0_index; \
    	tlbwi; \
    	li s2, entryhi; \
    	mtc0 s2, c0_entryhi; \
    	li s2, entrylo0; \
    	mtc0 s2, c0_entrylo0; \
    	li s2, entrylo1; \
    	mtc0 s2, c0_entrylo1; \
    	li s2, index; \
    	mtc0 s2, c0_index; \
    	tlbwi; \
    	li t1, data; \
    	li t0, base_addr; \
    	sw t1, 0x0(t0); \
    	lw s0, 0x0(t0); \
        li s2, data; \
        bne s0, s2, inst_error; \
        nop
    \end{lstlisting}

    \item {\bf TLBWR}:测试指令TLBWR。测试文件如下:

    \begin{lstlisting}[caption={n93\_tlbwr.S}, emph={TEST\_TLBWR}]
    LEAF(n93_tlbwr_test)
        .set noreorder
        lui  a0, 0x100
        li  v0, 0x0
    ###test inst
        TEST_TLBWR(0x0, 0x4002, 0x4042, 0x61754443, 0x000007a0)
        TEST_TLBWR(0x0, 0x4002, 0x4042, 0x5c4fb45a, 0x00000aac)
        TEST_TLBWR(0x0, 0x4002, 0x4042, 0x14908300, 0x00000ae8)
        TEST_TLBWR(0x0, 0x4002, 0x4042, 0x516da739, 0x000000cc)
        TEST_TLBWR(0x0, 0x4002, 0x4042, 0x85675a34, 0x00000510)
        TEST_TLBWR(0x0, 0x4002, 0x4042, 0x0e4dac98, 0x00000040)
        TEST_TLBWR(0x0, 0x4002, 0x4042, 0xd9c6eddb, 0x00000180)
        TEST_TLBWR(0x0, 0x4002, 0x4042, 0x5753dd01, 0x00000ca0)
        TEST_TLBWR(0x0, 0x4002, 0x4042, 0xe543b9f3, 0x0000031c)
        TEST_TLBWR(0x0, 0x4002, 0x4042, 0x4726aca2, 0x00000cf8)
        TEST_TLBWR(0x0, 0x4002, 0x4042, 0xb022040a, 0x00000800)
        TEST_TLBWR(0x0, 0x4002, 0x4042, 0x5ca0fd00, 0x00000834)
        TEST_TLBWR(0x0, 0x4002, 0x4042, 0x063ba000, 0x00000c64)
        TEST_TLBWR(0x0, 0x4002, 0x4042, 0xc2268cfe, 0x000001e8)
        TEST_TLBWR(0x0, 0x4002, 0x4042, 0x1611444c, 0x00000484)
        TEST_TLBWR(0x0, 0x4002, 0x4042, 0x33cc6f2a, 0x000001dc)
    ###detect exception
        bne v0, zero, inst_error
        nop
    ###score ++
        addiu s3, s3, 1
    ###output a0|s3
    inst_error:
        or t0, a0, s3
        sw t0, 0(s1)
        jr ra
        nop
    END(n93_tlbwr_test)
    \end{lstlisting}

    宏TEST\_TLBWR会首先调用TLBWI指令共15次以清空所有TLB表项;然后调用TLBWR指令写入其中一个表项。最终,通过先向基址处写入数据再读出,判断其是否与原数据相等,作为测试通过的依据:

    \begin{lstlisting}
    /* 93
     * TEST_TLBWR(ENTRYHI, ENTRYLO0, ENTRYLO1, data, base_addr)
     */
    #define TEST_TLBWR(entryhi, entrylo0, entrylo1, data, base_addr) \
    	li s2, 0; \
    	mtc0 s2, c0_entrylo0; \
    	mtc0 s2, c0_entrylo1; \
    	mtc0 s2, c0_entryhi; \
    	li s2, 0; \
    	mtc0 s2, c0_index; \
    	tlbwi; \
    	li s2, 1; \
    	mtc0 s2, c0_index; \
    	tlbwi; \
    	li s2, 2; \
    	mtc0 s2, c0_index; \
    	tlbwi; \
    	li s2, 3; \
    	mtc0 s2, c0_index; \
    	tlbwi; \
    	li s2, 4; \
    	mtc0 s2, c0_index; \
    	tlbwi; \
    	li s2, 5; \
    	mtc0 s2, c0_index; \
    	tlbwi; \
    	li s2, 6; \
    	mtc0 s2, c0_index; \
    	tlbwi; \
    	li s2, 7; \
    	mtc0 s2, c0_index; \
    	tlbwi; \
    	li s2, 8; \
    	mtc0 s2, c0_index; \
    	tlbwi; \
    	li s2, 9; \
    	mtc0 s2, c0_index; \
    	tlbwi; \
    	li s2, 10; \
    	mtc0 s2, c0_index; \
    	tlbwi; \
    	li s2, 11; \
    	mtc0 s2, c0_index; \
    	tlbwi; \
    	li s2, 12; \
    	mtc0 s2, c0_index; \
    	tlbwi; \
    	li s2, 13; \
    	mtc0 s2, c0_index; \
    	tlbwi; \
    	li s2, 14; \
    	mtc0 s2, c0_index; \
    	tlbwi; \
    	li s2, 15; \
    	mtc0 s2, c0_index; \
    	tlbwi; \
    	li s2, entryhi; \
    	mtc0 s2, c0_entryhi; \
    	li s2, entrylo0; \
    	mtc0 s2, c0_entrylo0; \
    	li s2, entrylo1; \
    	mtc0 s2, c0_entrylo1; \
    	tlbwr; \
    	li t1, data; \
    	li t0, base_addr; \
    	sw t1, 0x0(t0); \
    	lw s0, 0x0(t0); \
        li s2, data; \
        bne s0, s2, inst_error; \
        nop
    \end{lstlisting}

\end{enumerate}

\section{测试方式}

\subsection{仿真阶段}

仿真阶段的主要目的在于首先借助EDA工具(Vivado)方便的仿真调试功能,找出显而易见、较为简单的bug,以减轻接下来在硬件上测试调试的负担。

这一阶段仅测试CPU位于FPGA内部的核心逻辑是否正确,需使用自己模拟的RAM模块和Flash模块。

\paragraph{测试顶层文件} 本阶段测试所用的顶层文件(最小SOPC)如下。其中相比上一阶段,增加了对Flash的控制接口,并且包括BaseRAM和ExtRAM共2块RAM的接口,更好地模拟了开发板的真实情况:

\begin{lstlisting}[language=verilog, caption={仿真测试顶层文件thinpad\_min\_sopc.v}]
`include "defines.v"
`timescale 1ns/1ps

module thinpad_min_sopc();

    wire [31:0] ram_data;
    wire [19:0] ram_addr;
    wire [3:0] ram_be_n;
    wire ram_ce_n;
    wire ram_we_n;
    wire [31:0] ext_ram_data;
    wire [19:0] ext_ram_addr;
    wire [3:0] ext_ram_be_n;
    wire ext_ram_ce_n;
    wire ext_ram_we_n;
    wire [5:0] touch_btn;
    wire [22:0] flash_addr;
    wire [15:0] flash_data;

    reg clk;
    reg clk_25;
    reg CLOCK_11059200;
    reg rst;

    initial begin
        clk = 1'b0;
        forever #10 clk = ~clk;
    end
    initial begin
        clk_25 = 1'b0;
        forever #20 clk_25 = ~clk_25;
    end
    initial begin
        CLOCK_11059200 = 1'b0;
        forever #45.211227 CLOCK_11059200 = ~CLOCK_11059200;
    end

    initial begin
        rst = 1'b1;
        #195 rst = 1'b0;
        #500000000 $stop;
    end

    assign touch_btn = {rst, 5'b00000};

    thinpad_top thinpad_top0(
        .clk_in(clk),
        .clk_uart_in(CLOCK_11059200),
        .touch_btn(touch_btn),
        .base_ram_data(ram_data),
        .base_ram_addr(ram_addr),
        .base_ram_be_n(ram_be_n),
        .base_ram_we_n(ram_we_n),
        .base_ram_ce_n(ram_ce_n),
        .ext_ram_data(ext_ram_data),
        .ext_ram_addr(ext_ram_addr),
        .ext_ram_be_n(ext_ram_be_n),
        .ext_ram_we_n(ext_ram_we_n),
        .ext_ram_ce_n(ext_ram_ce_n),
        .flash_data(flash_data),
        .flash_a(flash_addr)

    );

    flash flash0(
        .clk(clk),
        .a(flash_addr[22:1]),
        .data(flash_data)
    );

    ram ram0(
        .clk(clk),
        .we(ram_we_n),
        .addr(ram_addr),
        .be(ram_be_n),
        .data(ram_data),
        .ce(ram_ce_n)
    );

    ram ram1(
        .clk(clk),
        .we(ext_ram_we_n),
        .addr(ext_ram_addr),
        .be(ext_ram_be_n),
        .data(ext_ram_data),
        .ce(ext_ram_ce_n)
    );

endmodule // openmips_min_sopc
\end{lstlisting}


\paragraph{模拟RAM}
模拟的RAM模块需首先将功能测例编译后的二进制文件转换成Verilog可接受的ASC-II文本文件,而后在initial语句中导入该文件,将CPU指令计数器置为0x80000000,开始仿真运行:

\begin{lstlisting}[language=verilog, caption={模拟RAM ram.v}]
module ram(
    input wire clk,
    input wire ce,
    input wire we,
    input wire oe,
    input wire[19:0] addr,
    input wire[3:0] be,
    inout wire[31:0] data
);
    reg[31:0] data_ram[0:1048575];

    initial $readmemh ("ram.data", data_ram);

    assign oe = 1'b0;

    always @(posedge clk) begin
        if (ce == 1'b1) begin

        end else if (we == 1'b0) begin
            if (be[3] == 1'b0) begin
                data_ram[addr][7:0] <= data[31:24];
            end
            if (be[2] == 1'b0) begin
                data_ram[addr][15:8] <= data[23:16];
            end
            if (be[1] == 1'b0) begin
                data_ram[addr][23:16] <= data[15:8];
            end
            if (be[0] == 1'b0) begin
                data_ram[addr][31:24] <= data[7:0];
            end
        end
    end

    assign data = we? { data_ram[addr][7:0], data_ram[addr][15:8], data_ram[addr][23:16], data_ram[addr][31:24] }: 32'bzzzzzzzzzzzzzzzzzzzzzzzzzzzzzzzz;

endmodule // data_ram
\end{lstlisting}


\paragraph{模拟Flash}
模拟的Flash与模拟RAM类似,同样在其initial块中导入数据:

\begin{lstlisting}
module flash(
    input wire clk,
    input wire ce,
    input wire we,
    input wire oe,
    input wire rp,
    input wire byte,
    input wire[21:0] a,
    inout wire[15:0] data
);
    reg[15:0] data_flash[0:4194303];

    initial $readmemh ("flash.data", data_flash);

    assign rp = 1'b1;
    assign byte = 1'b1;
    assign we = 1'b1;
    assign oe = 1'b0;
    assign ce = 1'b0;

    assign data = { data_flash[a][7:0], data_flash[a][15:8] };

endmodule // data_flash
\end{lstlisting}


\begin{lstlisting}[language=verilog, caption={ram.v的initial块}]
     initial $readmemh ("ram.data", data_ram);
\end{lstlisting}

通过阅读功能测例的代码我们可知,19号寄存器的数值存放了功能测例的通过条数。因此,只需在仿真中观察该寄存器的值是否随着每个测试点的结束自增1即可。

\subsection{硬件阶段}

硬件阶段的主要目的在于在真实开发板上运行功能测例,加入对RAM部分的测试,并找出CPU可能存在的时序漏洞。

在这一阶段,需将功能测例定义的七段数码管地址通过MMU映射至开发板的七段数码管,将编译后的功能测例烧入BaseRAM中,则七段数码管显示的即为测试点的通过个数。

\begin{lstlisting}[language=c, caption={start.S中定义的LED与七段数码管地址}]
    #define LED_ADDR  0xbfd0f000    // LED virtual address
    #define NUM_ADDR  0xbfd0f010    // 7-segments digital display virtual address
\end{lstlisting}

\section{测试结果}

77条功能测例全部通过,见以下表格(其中硬件测试时钟频率为25MHz,``P''-通过,``N''-无需实现)。

\testtable{序号}{测试程序}{功能测试点}{仿真测试}{硬件测试}
1    & ADD   & 执行 ADD 指令是否产生正确的运算结果(未测试整型溢出例外的情况)  & P    & P       \\
2    & ADDI  & 执行 ADDI 指令是否产生正确的运算结果(未测试整型溢出例外的情况) & P    & P       \\
3    & ADDU  & 执行 ADDU 指令是否产生正确的运算结果               & P    & P       \\
4    & ADDIU & 执行 ADDIU 指令是否产生正确的运算结果              & P    & P       \\
5    & SUB   & 执行 SUB 指令是否产生正确的运算结果(未测试整型溢出例外的情况)  & P    & P       \\
6    & SUBU  & 执行 SUBU 指令是否产生正确的运算结果               & P    & P       \\
7    & SLT   & 执行 SLT 指令是否产生正确的运算结果                & P    & P       \\
8    & SLTI  & 执行 SLTI 指令是否产生正确的运算结果               & P    & P       \\
9    & SLTU  & 执行 SLTU 指令是否产生正确的运算结果               & P    & P       \\
10   & SLTIU & 执行 SLTIU 指令是否产生正确的运算结果              & P    & P       \\
11   & DIV   & 执行 DIV 指令是否产生正确的运算结果                & N    & N       \\
12   & DIVU  & 执行 DIVU 指令是否产生正确的运算结果               & N    & N       \\
13   & MULT  & 执行 MULT 指令是否产生正确的运算结果               & P    & P       \\
14   & MULTU & 执行 MULTU 指令是否产生正确的运算结果              & P    & P       \\
\testtableend{算术运算指令(共12(0xC)条)}

\testtable{序号}{测试程序}{功能测试点}{仿真测试}{硬件测试}
15   & AND  & 执行 AND 指令是否产生正确的运算结果  & P    & P       \\
16   & ANDI & 执行 ANDI 指令是否产生正确的运算结果 & P    & P       \\
17   & LUI  & 执行 LUI 指令是否产生正确的运算结果  & P    & P       \\
18   & NOR  & 执行 NOR 指令是否产生正确的运算结果  & P    & P       \\
19   & OR   & 执行 OR 指令是否产生正确的运算结果   & P    & P       \\
20   & ORI  & 执行 ORI 指令是否产生正确的运算结果  & P    & P       \\
21   & XOR  & 执行 XOR 指令是否产生正确的运算结果  & P    & P       \\
22   & XORI & 执行 XORI 指令是否产生正确的运算结果 & P    & P       \\
23   & SLLV & 执行 SLLV 指令是否产生正确的移位结果 & P    & P       \\
24   & SLL  & 执行 SLL 指令是否产生正确的移位结果  & P    & P       \\
25   & SRAV & 执行 SRAV 指令是否产生正确的移位结果 & P    & P       \\
26   & SRA  & 执行 SRA 指令是否产生正确的移位结果  & P    & P       \\
27   & SRLV & 执行 SRLV 指令是否产生正确的移位结果 & P    & P       \\
28   & SRL  & 执行 SRL 指令是否产生正确的移位结果  & P    & P       \\
\testtableend{逻辑运算指令(共14(0xE)条)}

\testtable{序号}{测试程序}{功能测试点}{仿真测试}{硬件测试}
29   & BEQ    & 执行 BEQ 指令是否产生正确的判断和跳转结果(延迟槽指令为nop,未测试延迟槽) & P    & P       \\
30   & BNE    & 执行 BNE 指令是否产生正确的判断和跳转结果(延迟槽指令为nop,未测试延迟槽) & P    & P       \\
31   & BGEZ   & 执行 BGEZ 指令是否产生正确的判断和跳转结果(延迟槽指令为 nop,未测试延迟槽) & P    & P       \\
32   & BGTZ   & 执行 BGTZ 指令是否产生正确的判断和跳转结果(延迟槽指令为 nop,未测试延迟槽) & P    & P       \\
33   & BLEZ   & 执行 BLEZ 指令是否产生正确的判断和跳转结果(延迟槽指令为 nop,未测试延迟槽) & P    & P       \\
34   & BLTZ   & 执行 BLTZ 指令是否产生正确的判断和跳转结果(延迟槽指令为 nop,未测试延迟槽) & P    & P       \\
35   & BGEZAL & 执行 BGEZAL 指令是否产生正确的判断、跳转和链接结果(延迟槽指令为 nop,未测试延迟槽) & N    & N       \\
36   & BLTZAL & 执行 BLTZAL 指令是否产生正确的判断、跳转和链接结果(延迟槽指令为 nop,未测试延迟槽) & N    & N       \\
37   & J      & 执行 J 指令是否产生正确的跳转结果(延迟槽指令为 nop,未测试延迟槽)    & P    & P       \\
38   & JAL    & 执行 JAL 指令是否产生正确的跳转和链接结果(延迟槽指令为nop,未测试延迟槽) & P    & P       \\
39   & JR     & 执行 JR 指令是否产生正确的跳转结果(延迟槽指令为 nop,未测试延迟槽)   & P    & P       \\
40   & JALR   & 执行 JALR 指令是否产生正确的跳转和链接结果(延迟槽指令为 nop,未测试延迟槽) & P    & P       \\
\testtableend{分支跳转指令(共10(0xA)条)}

\testtable{序号}{测试程序}{功能测试点}{仿真测试}{硬件测试}
41   & MFHI & 执行 MTHI 指令是否正确地将寄存器值写入 HI 寄存器,执行MFHI 指令是否正确地读出 HI 寄存器的值到寄存器 & P    & P       \\
42   & MFLO & 执行 MTLO 指令是否正确地将寄存器值写入 LO 寄存器,执行MFLO 指令是否正确地读出 LO 寄存器的值到寄存器 & P    & P       \\
43   & MTHI & 执行 MTHI 指令是否正确地将寄存器值写入 HI 寄存器,执行MFHI 指令是否正确地读出 HI 寄存器的值到寄存器 & P    & P       \\
44   & MTLO & 执行 MTLO 指令是否正确地将寄存器值写入 HI 寄存器,执行MFLO 指令是否正确地读出 HI 寄存器的值到寄存器 & P    & P       \\
\testtableend{HILO寄存器指令(共4(0x4)条)}

\testtable{序号}{测试程序}{功能测试点}{仿真测试}{硬件测试}
45   & BREAK   & 执行 BREAK 指令是否正确地产生断点例外                   & N    & N       \\
46   & SYSCALL & 执行 SYSCALL 指令是否正确地产生系统调用例外               & P    & P       \\
55   & ERET    & 执行 ERET 指令是否正确地从中断、例外处理程序返回              & P    & P       \\
56   & MFC0    & 执行 MTC0 指令是否正确地将寄存器值写入目的 CP0 寄存器,执行 MFC0 指令是否正确地读出源 CP0 寄存器的值到寄存器 & P    & P       \\
57   & MTC0    & 执行 MTC0 指令是否正确地将寄存器值写入目的 CP0 寄存器,执行 MFC0 指令是否正确地读出源 CP0 寄存器的值到寄存器 & P    & P       \\
\testtableend{CP0指令(共4(0x4)条)}

\testtable{序号}{测试程序}{功能测试点}{仿真测试}{硬件测试}
47   & LB   & 结合 SW 指令,执行 LB 指令是否产生正确的访存结果  & P    & P       \\
48   & LBU  & 结合 SW 指令,执行 LBU 指令是否产生正确的访存结果 & P    & P       \\
49   & LH   & 结合 SW 指令,执行 LH 指令是否产生正确的访存结果  & P    & P       \\
50   & LHU  & 结合 SW 指令,执行 LHU 指令是否产生正确的访存结果 & P    & P       \\
51   & LW   & 结合 SW 指令,执行 LW 指令是否产生正确的访存结果  & P    & P       \\
52   & SB   & 结合 LW 指令,执行 SB 指令是否产生正确的访存结果  & P    & P       \\
53   & SH   & 结合 LW 指令,执行 SH 指令是否产生正确的访存结果  & P    & P       \\
54   & SW   & 结合 LW 指令,执行 SW 指令是否产生正确的访存结果  & P    & P       \\
\testtableend{访存指令(共8(0x8)条)}

\testtable{序号}{测试程序}{功能测试点}{仿真测试}{硬件测试}
58   & ADD\_EX                   & 测试 ADD 指令整型溢出例外    & P    & P       \\
59   & ADDI\_EX                  & 测试 ADDI 指令整型溢出例外   & P    & P       \\
60   & SUB\_EX                   & 测试 SUB 指令整型溢出例外    & P    & P       \\
61   & LH\_EX                    & 测试 LH 指令访存地址非对齐例外  & N    & N       \\
62   & LHU\_EX                   & 测试 LHU 指令访存地址非对齐例外 & N    & N       \\
63   & LW\_EX                    & 测试 LW 指令访存地址非对齐例外  & N    & N       \\
64   & SH\_EX                    & 测试 SH 指令访存地址非对齐例外  & N    & N       \\
65   & SW\_EX                    & 测试 SW 指令访存地址非对齐例外  & N    & N       \\
66   & ERET\_EX                  & 测试取指地址非对齐例外        & N    & N       \\
67   & RESERVED\_- & 测试保留指令例外           & N    & N       \\
& INSTRUCTION\_EX & & & \\
80   & BEQ\_EX\_DS               & 测试延迟槽例外            & P    & P       \\
81   & BNE\_EX\_DS               & 测试延迟槽例外            & P    & P       \\
82   & BGEZ\_EX\_DS              & 测试延迟槽例外            & P    & P       \\
83   & BGTZ\_EX\_DS              & 测试延迟槽例外            & P    & P       \\
84   & BLEZ\_EX\_DS              & 测试延迟槽例外            & P    & P       \\
85   & BLTZ\_EX\_DS              & 测试延迟槽例外            & P    & P       \\
86   & BGEZAL\_EX\_DS            & 测试延迟槽例外            & N    & N       \\
87   & BLTZAL\_EX\_DS            & 测试延迟槽例外            & N    & N       \\
88   & J\_EX\_DS                 & 测试延迟槽例外            & P    & P       \\
89   & JAL\_EX\_DS               & 测试延迟槽例外            & P    & P       \\
90   & JR\_EX\_DS                & 测试延迟槽例外            & P    & P       \\
91   & JALR\_EX\_DS              & 测试延迟槽例外            & P    & P       \\
\testtableend{异常指令(共13(0xD)条)}

\testtable{序号}{测试程序}{功能测试点}{仿真测试}{硬件测试}
68   & BEQ\_DS    & 测试延迟槽 & P    & P       \\
69   & BNE\_DS    & 测试延迟槽 & P    & P       \\
70   & BGEZ\_DS   & 测试延迟槽 & P    & P       \\
71   & BGTZ\_DS   & 测试延迟槽 & P    & P       \\
72   & BLEZ\_DS   & 测试延迟槽 & P    & P       \\
73   & BLTZ\_DS   & 测试延迟槽 & P    & P       \\
74   & BGEZAL\_DS & 测试延迟槽 & N    & N       \\
75   & BLTZAL\_DS & 测试延迟槽 & N    & N       \\
76   & J\_DS      & 测试延迟槽 & P    & P       \\
77   & JAL\_DS    & 测试延迟槽 & P    & P       \\
78   & JR\_DS     & 测试延迟槽 & P    & P       \\
79   & JALR\_DS   & 测试延迟槽 & P    & P       \\
\testtableend{延迟槽指令(共10(0xA)条)}

\section{问题与解决}

% 功能测例自己会破坏自己,不能一次测太多。
在测试中我们发现,功能测例会通过load指令写入自身的代码区,导致自身逻辑遭到破坏。

例如,下面的第一条测试LB指令的代码,会将数据写入VA (0x800244e8 + 0x00002174) = 0x8002665C处:

\begin{lstlisting}[caption={n47\_lb.S中会破坏自身程序的访存指令}]
...
# *(0x800244e8 + 0x00002174) = 0xe83814f0
TEST_LB(0xe83814f0, 0x800244e8, 0x00002174, 0x00002174, 0xfffffff0)
TEST_LB(0xbb62f9ba, 0x80021408, 0x00003c40, 0x00003c40, 0xffffffba)
TEST_LB(0x9eb52b80, 0x8002d46c, 0x000002ae, 0x000002ac, 0xffffffb5)
...
\end{lstlisting}

然而,由于功能测例的初始地址在VA 0x80000000处,当测试范围为整个功能测例的所有代码时,会导致该地址处实际属于功能测例自身的代码段:

\begin{lstlisting}[emph={8002665c}, caption={功能测例在objdump后的代码片段}]
...
80026654:	10000005 	b	8002666c <n40_jalr_test+0x954c>
80026658:	00000000 	nop
8002665c:	00000021 	move	zero,zero
80026660:	00000021 	move	zero,zero
80026664:	3c12d65a 	lui	s2,0xd65a
...
\end{lstlisting}

从而,要想通过全部功能测例,有以下2种解决方案:

\begin{enumerate}
    \item {\bf 使用2块RAM}:同时使用BaseRAM和ExtRAM,分别用于存储指令和数据。仿真测试显示这样可以通过所有功能测例而不会触发上述bug。
    \item {\bf 分段测试}:将功能测例逐段进行测试(其余部分注释掉),这样功能测例代码量减少,不会蔓延到被写入的内存区域。最终笔者得以用这样的方式通过全部功能测例。
\end{enumerate}
