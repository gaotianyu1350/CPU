\chapter{初期指令测试}

\section{简介概述}

在开发初期,流水线仅具雏形,外部设备仅包括使用Verilog模拟的ROM与RAM,分别用于存储指令与数据。

这一阶段的测试因而主要通过Test Bench技术实现,通过自行编写的编译工具将MIPS 32汇编指令转为机器代码,并在ROM模块的initial块内读入。仿真测试时,CPU从ROM模块开始执行测试指令。

初期测试的框架如下:

\image[5in]{testbench}{初期测试框架}

后期在此基础上增加了RAM模块,用于测试访存指令的正确性。


\section{测试范畴}

\subsection{测试覆盖面}

\image[5in]{simpletest}{初期指令测试范畴(绿-可以测试;灰-不能测试)}

初期指令测试主要测试了CPU以下部件是否实现正确:

\begin{enumerate}
    \item {\bf 五级流水线}:包括IF、ID、EX、MEM、WB共5个模块。
    \item {\bf 寄存器}:包括寄存器堆(32个通用寄存器)、HI/LO寄存器共2个模块。
    \item {\bf Control}:包括Control(流水线控制器)共1个模块。
\end{enumerate}

初期指令测试不能全面测试的CPU部件包括:

\begin{enumerate}
    \item {\bf CP0}:初期指令测试完全不考虑异常处理,故不能测试CP0。
    \item {\bf MMU}:初期指令测试完全不考虑地址映射,更不包括TLB,故不能测试MMU。
    \item {\bf ROM}:初期指令测试仅使用模拟的ROM,无BootLoader过程,故不能测试ROM。
    \item {\bf RAM}:初期指令测试仅使用模拟的RAM,故不能测试RAM。
    \item {\bf Flash}:初期指令测试完全不考虑Flash,故不能测试Flash。
    \item {\bf 串口}:初期指令测试完全通过仿真验证,不需要串口通信,故不能测试串口。
\end{enumerate}

\subsection{测试要点}

初期指令测试的测试要点包括如下几个部分:

\begin{enumerate}
    \item {\bf 算术运算指令}:共9条指令,包括ADDIU、ADDU、SUBU、MULT、SLT、SLTI、SLTIU、SLTU、LUI。
    \item {\bf 逻辑运算指令}:共13条指令,包括AND、ANDI、OR、ORI、NOR、XOR、XORI、SLL、SLLV、SRL、SRLV、SRA、SRAV。
    \item {\bf 分支跳转指令}:共10条指令,包括BEQ、BNE、BGEZ、BGTZ、BLEZ、BLTZ、J、JR、JAL、JALR。
    \item {\bf HI/LO寄存器指令}:共4条指令,包括MFHI、MFLO、MTHI、MTLO。
    \item {\bf 访存指令}:共8条指令,包括LB、LBU、LW、SB、SW。
\end{enumerate}

\section{测试方法}

\subsection{TestBench文件}
本阶段测试基于如下TestBench文件:

\begin{lstlisting}[language=verilog, caption={TestBench文件}]
`include "defines.v"
`timescale 1ns/1ps

module openmips_min_sopc_tb();

    reg CLOCK_50;
    reg rst;

    initial begin
        CLOCK_50 = 1'b0;
        forever #10 CLOCK_50 = ~CLOCK_50;
    end

    initial begin
        rst = `RstEnable;
        #195 rst = `RstDisable;
        #5000000 $stop;
    end

    openmips_min_sopc openmips_min_sopc0(
        .clk(CLOCK_50),
        .rst(rst)
    );

endmodule // openmips_min_sopc_tb()
\end{lstlisting}

\subsection{最小SOPC}
其中的openmips\_min\_sopc0模块为五级流水线的顶层模块(最小SOPC,System-on-a-Programmable-Chip),其中集成了对ROM与RAM的控制逻辑。TestBench为其提供时钟与复位信号:

\begin{lstlisting}[language=verilog, caption={OpenMIPS 最小SOPC}]
`include "defines.v"

module openmips_min_sopc(
    input wire clk,
    input wire rst
);

    wire[`InstAddrBus] inst_addr;
    wire[`InstBus] inst;
    wire rom_ce;
    wire mem_we_i;
    wire[`RegBus] mem_addr_i;
    wire[`RegBus] mem_data_i;
    wire[`RegBus] mem_data_o;
    wire[3:0] mem_sel_i;
    wire mem_ce_i;

    // Exception
    wire[5:0] int;
    wire timer_int;

    assign int = {5'b00000, timer_int};

    openmips openmips0(
        .clk(clk),
        .rst(rst),
        .if_addr_o(inst_addr),
        .if_data_i(inst),
        .if_ce_o(rom_ce),
        .mem_we_o(mem_we_i),
        .mem_addr_o(mem_addr_i),
        .mem_sel_o(mem_sel_i),
        .mem_data_o(mem_data_i),
        .mem_data_i(mem_data_o),
        .mem_ce_o(mem_ce_i),
        .int_i      (int),
        .timer_int_o(timer_int)
    );

    inst_rom inst_rom0(
        .addr(inst_addr),
        .inst(inst),
        .ce(rom_ce)
    );

    data_ram data_ram0(
        .clk(clk),
        .we(mem_we_i),
        .addr(mem_addr_i),
        .sel(mem_sel_i),
        .data_i(mem_data_i),
        .data_o(mem_data_o),
        .ce(mem_ce_i)
    );

endmodule // openmips_min_sopc
\end{lstlisting}

\subsection{MIPS测试文件}

% <TODO>: 整理所有测例 无需整理

本部分的指令测例大部分修改自《自己动手写CPU》一书中提供的功能测例。

例如,第一条指令ORI的测试文件如下(各条指令的期望结果如后方注释,用于仿真时与波形图比对)
\begin{lstlisting}[caption={ORI指令测试文件}]
.org 0x0
.global _start
.set noat
_start:
    ori $1,$0,0x1100        # $1 = $0 | 0x1100 = 0x1100
    ori $1,$1,0x0020        # $1 = $1 | 0x0020 = 0x1120
    ori $1,$1,0x4400        # $1 = $1 | 0x4400 = 0x5520
    ori $1,$1,0x0044        # $1 = $1 | 0x0044 = 0x5564
\end{lstlisting}

\subsection{MIPS编译工具}
由于开发初期尚未涉及操作系统的编译,而mips-gcc工具链配置较为繁琐,故笔者自行实现了MIPS编译工具,用于将MIPS 32汇编代码转化为Verilog的readmemh函数可接受的ASC-II文件。

编译工具采用Python语言编写,支持了本阶段多数指令(包括分支跳转指令)的转化工作,方便了本阶段的仿真调试工作。

\section{测试结果}

经过测试,本阶段涉及的指令测试点均能通过。

需注意,本阶段使用的测例均较为简单,每条测例使用的数据也不多。这一不足由下一阶段的功能测例得到了弥补。
