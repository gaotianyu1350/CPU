\chapter{经验总结}

    在开发过程中,我们证明了以下方式确实非常值得借鉴:

    \begin{enumerate}
        \itembf{统一环境}: 我们的Windows和Vivado版本号全部一致,Ubuntu采用云服务器,高度一致的环境使得我们不必因为环境的问题而产生莫名其妙的bug,其他组成员便存在因为编译器版本号等问题导致了结果不同,浪费了大量时间。
        \itembf{共同学习}: 所有人都阅读了相关文档与《自己动手写CPU》大部分章节,对系统都有一个明确而清晰的认知。
        \itembf{沟通交流}: 每一部分完成后,一定要结合着代码,给其他成员讲清楚你在做什么,怎么实现的这个逻辑,让所有人对系统都有所了解。
        \itembf{及时测试}: 完成每一个功能后,一定要及时测试,即便是最简单的单条指令仿真测试,也能发现很多的问题,一定不要最后堆在一起进行测试,那样会造成成指数增长的调试时间。正是因为我们进行了严格的测试,因此最后调试时间大大缩减。
        \itembf{遵从开发规范}:每周更新Milestone,每一个任务都单独建立相应的Git Issue,建立对应分支,完成后发起合并请求,所有人review代码,遵从开发规范能保证版本控制得当,责任具体,划分明确,更可为今后文档撰写提供足够的参考。
    \end{enumerate}

    当然,我们在开发过程中也有一些不足,例如有时候因为别的任务而不得不改变进度,例如第12周由于因为作业过多我们本该调试通过的监控程序没有调过,但是这也正是软工“变”的意义所在,拥抱变化,及时调整策略,及时沟通,共同努力,明确任务。





