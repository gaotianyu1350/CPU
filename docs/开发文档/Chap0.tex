\chapter{文档说明}

本文档是NonExist组软工计原联合项目的开发文档。

当初次接到这个项目时,面对卷帙浩繁的文档,我们陷入了一种深深的迷茫,本文档的目的便是描述我们是如何一步一步理清需求,明确分工,最终造出一台32位计算机的。

项目从学期第2周开始,持续到第16周结束,共计15周。

查看GitLab的Milestone功能,我们可以看到每周的MileStone如下:
    
    \begin{enumerate}
        \itembf{Week 5}: Build CPU baseline
        \itembf{Week 6}: Add instructions
        \itembf{Week 7}: Complete all instructions, build ucore
        \itembf{Week 8}: Complete exception, learn external devices, read books
        \itembf{Week 9}: TLB/MMU/External Devices Required
        \itembf{Week 10}: Link MMU, external devices
        \itembf{Week 11}: Run function test on chips and debug
        \itembf{Week 12}: Run monitor on chip and debug
        \itembf{Week 13}: Run monitor on chip and ucore if possible
        \itembf{Week 14}: Run ucore on chip
        \itembf{Week 15}: More external devices and unit test
        \itembf{Week 16 later}: Document Required
    \end{enumerate}

通过观察Milestone,我们将整个项目周期划分为5个Sprint,每个Sprint三星期,其主线可以如下概括:

    \begin{enumerate}
        \itembf{Sprint 1}:Week 2 - Week 4 查阅相关文献
        \itembf{Sprint 2}:Week 5 - Week 7 指令流水、数据通路
        \itembf{Sprint 3}:Week 8 - Week 10 CP0、异常、MMU、外设连接
        \itembf{Sprint 4}:Week 11 - Week 13 调试功能测例、监控程序
        \itembf{Sprint 5}:Week 14 - Week 16 调试ucore、撰写文档
    \end{enumerate}

当然每个部分还做了一些额外的事情,比如配置开发环境,制作针对TLB的功能测例等等,在之后的部分会具体指出。

希望本文档能给读者带来裨益。
